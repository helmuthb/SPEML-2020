% !TeX spellcheck = en_US
\documentclass[sigconf,nonacm]{acmart}
\settopmatter{printacmref=false}
\pagestyle{empty}
\AtBeginDocument{%
  \providecommand\BibTeX{{%
    \normalfont B\kern-0.5em{\scshape i\kern-0.25em b}\kern-0.8em\TeX}}}
\copyrightyear{2020}
\acmYear{2020}
\setcopyright{rightsretained}
\begin{document}
\title{Security, Privacy \& Explainability in Machine Learning}
\subtitle{Exercise 1: Record-Linkage Attack}
\author{Thomas Jirout}
\email{thomas.jirout@tuwien.ac.at}
\affiliation{Mat.Nr. 01525606}
\author{Helmuth Breitenfellner}
\email{helmuth.breitenfellner@student.tuwien.ac.at}
\affiliation{Mat.Nr. 08725866}
\begin{abstract}
In this task we have been working on record-linkage attack.
For this we chose the dataset DBLP-Scholar provided by the University Leipzig\cite{DataSets},\cite{kopcke2010evaluation}
and used the Python Record Linkage Toolkit\cite{de_bruin_j_2019_3559043}.

This report summarizes the identification of sensitive attributes, attributes
used for linking, and the performance of the record-linkage attack.
\end{abstract}
\keywords{Security, Privacy, Machine Learning, Record Linkage, Python}
\maketitle
\section{Task description}

When performing a record linkage attack, there are a couple of steps
to be done:
\begin{itemize}
\item Acquire datasets
\item Cleanup \& pre-process data
\item Identify candidate pairs for linkage
\item Calculate matches on field-levels
\item Classify linked records
\item Evaluate classification
\end{itemize}

\subsection{Acquire datasets}

We were using the evaluation datasets, provided by the University Leipzig.
These datasets are already harmonized, i.e. the columns have the same names,
and there is also a set of ``ground truth'' linkage information
which we were using for the evaluation.

\subsection{Cleanup \& pre-processing}

Datasets contain errors: wrongly encoded strings, typos, or
incorrect years.
The purpose of the cleanup step is to reduce the influence of these
errors.
Approaches available include:
\begin{itemize}
\item Fuzzy text encoding, e.g. ``soundex'' code
\item Harmonization, e.g. lowercase of all text, removal of non-text characters
\end{itemize}

\subsection{Identify candidate pairs}

For performing a record linkage attack, one would have to compare each
record of the first with each record of the second dataset.
This becomes quickly infeasible once datasets reach hundred thousands
of records.

One approach to reduce this effort is using ``blocking''.
Here attributes are looked for which \emph{have} to agree for a
match to be considered.
Possible choice could be years where the errors are assumed to be
less prominent.

A variant of blocking is using a neighborhood. Instead of only
considering candidates with the same value one allows a small
difference of the values between the datasets.

Both blocking and neighborhoods significantly reduce the number
of candidate pairs to be inspected.

\subsection{Calculate matches on field-level}

\subsection{Classify linked records}

\subsection{Evaluate classification}

\section{Datasets}

We used the following two datasets:
\begin{description}
\item[DBLP:] Publication records from the Digital Bibliography \& Library Project
\item[Scholar:] Publication records from Google Scholar
\end{description}

\subsection{Sensitive attributes}

Normally datasets of articles do not contain sensitive information,
and authors would like to have the contained data shared with
as many people as possible.

For the purpose of out evaluation we were qualifying the
ID on Google Scholar as sensitive.

The idea is that through these ID values more information
can potentially be revealed, e.g. citations of the authors.

\subsection{Attributes used for linking}

We were using the following attributes for linking:
\begin{description}
\item[Name:] Name of the publication
\item[Venue:] Venue of the publication
\item[Authors:] Names of the authors
\end{description}

\section{Evaluation \& Summary}

For the evaluation we looked at the \emph{F1-score} as a unified
measure of both recall and precision.

\bibliographystyle{ACM-Reference-Format}
\bibliography{report}
\end{document}
