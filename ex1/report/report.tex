% !TeX spellcheck = en_US
\documentclass[sigconf,nonacm]{acmart}
\settopmatter{printacmref=false}
\pagestyle{empty}
\AtBeginDocument{%
  \providecommand\BibTeX{{%
    \normalfont B\kern-0.5em{\scshape i\kern-0.25em b}\kern-0.8em\TeX}}}
\copyrightyear{2020}
\acmYear{2020}
\setcopyright{rightsretained}
\begin{document}
\title{Security, Privacy \& Explainability in Machine Learning}
\subtitle{Exercise 1: Record-Linkage Attack}
\author{Thomas Jirout}
\email{thomas.jirout@tuwien.ac.at}
\affiliation{Mat.Nr. 01525606}
\author{Helmuth Breitenfellner}
\email{helmuth.breitenfellner@student.tuwien.ac.at}
\affiliation{Mat.Nr. 08725866}
\begin{abstract}
In this task we have been working on record-linkage attack.
For this we chose the dataset DBLP-Scholar provided by the University Leipzig\cite{DataSets},\cite{kopcke2010evaluation}
and used the Python Record Linkage Toolkit\cite{de_bruin_j_2019_3559043}.

This report summarizes the identification of sensitive attributes, attributes
used for linking, and the performance of the record-linkage attack.
\end{abstract}
\keywords{Security, Privacy, Machine Learning, Record Linkage, Python}
\maketitle
\section{Task description}
\section{Datasets}
\subsection{Sensitive attributes}
\subsection{Attributes used for linking}
\section{Evaluation \& Summary}
\bibliographystyle{ACM-Reference-Format}
\bibliography{report}
\end{document}
